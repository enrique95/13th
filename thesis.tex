\documentclass[12pt]{report} 
\usepackage[
a4paper,
vmargin=2.5cm,
hmargin=3cm
]{geometry}
\renewcommand{\baselinestretch}{1.5}
\parskip=6pt
\usepackage[table]{xcolor}
\definecolor{azulUC3M}{RGB}{0,0,102}
\definecolor{gray97}{gray}{.97}
\definecolor{gray75}{gray}{.75}
\definecolor{gray45}{gray}{.45}
\usepackage[a-1b]{pdfx}
\usepackage{hyperref}
\hypersetup{colorlinks=true,
	linkcolor=black, 
	urlcolor=blue} 
\usepackage{amsmath,amssymb,amsfonts,amsthm}
\usepackage{txfonts} 
\usepackage[T1]{fontenc}
\usepackage[utf8]{inputenc}
\usepackage[english]{babel} 
\usepackage[babel, english=american]{csquotes}
\AtBeginEnvironment{quote}{\small}
\usepackage{fancyhdr}
\pagestyle{fancy}
\fancyhf{}
\renewcommand{\headrulewidth}{0pt}
\rfoot{\thepage}
\fancypagestyle{plain}{\pagestyle{fancy}}
\usepackage{titlesec}
\usepackage{titletoc}
\titleformat{\chapter}[block]
{\large\bfseries\filcenter}
{\thechapter.}
{5pt}
{\MakeUppercase}
{}
\titlespacing{\chapter}{0pt}{0pt}{*3}
\titlecontents{chapter}
[0pt]                                               
{}
{\contentsmargin{0pt}\thecontentslabel.\enspace\uppercase}
{\contentsmargin{0pt}\uppercase}                        
{\titlerule*[.7pc]{.}\contentspage}                 
\titleformat{\section}
{\bfseries}
{\thesection.}
{5pt}
{}
\titlecontents{section}
[5pt]                                               
{}
{\contentsmargin{0pt}\thecontentslabel.\enspace}
{\contentsmargin{0pt}}
{\titlerule*[.7pc]{.}\contentspage}
\titleformat{\subsection}
{\normalsize\bfseries}
{\thesubsection.}
{5pt}
{}
\titlecontents{subsection}
[10pt]                                               
{}
{\contentsmargin{0pt}                          
	\thecontentslabel.\enspace}
{\contentsmargin{0pt}}                        
{\titlerule*[.7pc]{.}\contentspage}  
\usepackage{multirow} 
\usepackage{caption} 
\usepackage{floatrow} 
\usepackage{array}
\newcolumntype{P}[1]{>{\centering\arraybackslash}p{#1}}
\DeclareCaptionFormat{upper}{#1#2\uppercase{#3}\par}
\captionsetup[table]{
	justification=raggedright,
	labelsep=period,
	labelfont=small,
	singlelinecheck=false,
	font={small,bf}
}
\usepackage{graphicx}
\graphicspath{{imagenes/}} 
\usepackage{subcaption}
\captionsetup[figure]{
	format=hang,
	name=Figure,
	singlelinecheck=off,
	labelsep=period,
	labelfont=small,
	font=small		
}
\usepackage{chngcntr} %para numeración contínua de las notas al pie
\counterwithout{footnote}{chapter}
\usepackage{listings}
\lstdefinestyle{estilo}{ frame=Ltb,
	framerule=0pt,
	aboveskip=0.5cm,
	framextopmargin=3pt,
	framexbottommargin=3pt,
	framexleftmargin=0.4cm,
	framesep=0pt,
	rulesep=.4pt,
	backgroundcolor=\color{gray97},
	rulesepcolor=\color{black},
	basicstyle=\ttfamily\footnotesize,
	keywordstyle=\bfseries,
	stringstyle=\ttfamily,
	showstringspaces = false,
	commentstyle=\color{gray45},     
	numbers=left,
	numbersep=15pt,
	numberstyle=\tiny,
	numberfirstline = false,
	breaklines=true,
	xleftmargin=\parindent
}

\captionsetup[lstlisting]{font=small, labelsep=period} 
\lstset{style=estilo}
\renewcommand{\lstlistingname}{\uppercase{Código}}
\usepackage[style=apa, backend=biber, natbib=true, hyperref=true, uniquelist=false, sortcites]{biblatex}
\DeclareLanguageMapping{spanish}{spanish-apa}

\addbibresource{bibliografia/bibliografia.bib} % llama al archivo bibliografia.bib que utilizamos de ejemplo




\begin{document}
\pagenumbering{roman} 	

\begin{titlepage}
	\begin{sffamily}
	\color{azulUC3M}
	\begin{center}
		\begin{figure}[H] 
			\makebox[\textwidth][c]{\includegraphics[width=16cm]{Portada_Logo.png}}
		\end{figure}
		\vspace{2.5cm}
		\begin{Large}
			University Degree in Economics\\			
			Academic Year 2018-2019\\
			\vspace{2cm}		
			\textsl{Bachelor Thesis}
			\bigskip
			
		\end{Large}
		 	{\Huge ``Abolition of Slavery in the US and its effects on Agricultural Output''}\\
		 	\vspace*{0.5cm}
	 		\rule{10.5cm}{0.1mm}\\
			\vspace*{0.9cm}
			{\LARGE Juan Enrique García González}\\ 
			\vspace*{1cm}
		\begin{Large}
			Pablo Martinelli Lasheras \\
			Madrid \today\\
		\end{Large}
	\end{center}
	\vfill
	\color{black}
	\includegraphics[width=4.2cm]{imagenes/creativecommons.png}\\  %incluimos el logotipo de creativecommons
	This work is licensed under Creative Commons \textbf{Attribution – Non Commercial – Non Derivatives}
	\end{sffamily}
\end{titlepage}

\newpage 
\thispagestyle{empty}
\mbox{}


\renewcommand\abstractname{\large\uppercase{Summary}}
\begin{abstract}
\thispagestyle{plain}
\setcounter{page}{3}
	
	
Slavery has been used widely as a cheap source of human labor since the beginning of history, being used intensively in the agricultural sector. 

Early modern age America has been going hand in hand with the history of slavery. In the case of the United States, it has been quite a controversial topic. Since the inception of the country, this issue had been widely debated. It was eventually banned in the Northern states, while the imports of slaves were restricted nationwide; although not completely stopped. This circumstance helped the Southern states to build an agricultural production based on slave workforce.

%Various advancements in technology made possible the use of cotton gin, which dramatically decreased the costs of cotton manufacturing. This circumstance contributed not only to the use of slaves in the production of tobacco but also to the massive production of cotton, being the area of the \textit{Cotton Belt} particularly intensive in the production of this crop.
%\par
Many policies were passed through the Congress to increase the pressure on slave ownership, until this conflict eventually developed in the Civil War. This war resulted in a victory for the Northern states, and thus the liberation of slaves ensued via the 13th Amendment. This law essentially meant the end of forced labor.

The aim of this research will be to analyze the evolution of agricultural output of the different states that existed in this time and the shocks of agricultural production after the abolition of slavery.
	
	\textbf{Keywords:}
	slavery, 13th Amendment, agricultural data, counties, USA, panel data, differences in differences model, historical data, abolitionism, 1860, 1870, econometrics
	\vfill
\end{abstract}
	\newpage 
	\thispagestyle{empty}
	\mbox{}



\chapter*{Dedication}

\setcounter{page}{5}
	

	I would like to thank my parents for their patient, continuous support throughout my career, both mentally and financially. I would like to also thank my colleague Iván Logrosán for his resourceful ideas as well as Pablo Martinelli for his meaningful tutoring. At last but no least I would like to thank the rest of Carlos III University staff that had helped me making this bachelor thesis possible, specially the library staff.
	\vfill
	
	\newpage
	\thispagestyle{empty}
	\mbox{}
	


\tableofcontents
\thispagestyle{fancy}

\newpage % página en blanco o de cortesía
\thispagestyle{empty}
\mbox{}

\listoffigures
\thispagestyle{fancy}

\newpage 
\thispagestyle{empty}
\mbox{}


\listoftables
\thispagestyle{fancy}

\newpage 
\thispagestyle{empty}
\mbox{}


	
\clearpage
\pagenumbering{arabic} % numeración con múmeros arábigos para el resto de la publicación	

\chapter{Introduction}



\section{Motivation of work and goals}
The matter of slavery was one of the topics that contributed to the upsurge in the 70s of the study of economic history as a topic closer to the empirical approach. This impact on this field of economics and the importance of the event itself in history are the motivations of this study. The research will analyze the practice of slavery and its effects in a very crucial period: the decades before and after the passing in 1865 of the 13th Amendment, which determined the abolition of slavery in the United States of America.
\section{Literature review}
There are two main trends that deal with the different effects that had slavery on agricultural production. The first position states that slavery did have a positive effect, by using production systems with enslaved labor force that were more efficient than the family-based farms. One very important concept that belonged to slavery-specific production and is crucial to explain the efficency of slavery was the so-called "gang system". This system consisted of individual, well monitored processes, in which the slavers could easily allocate their enslaved workhand. Every gang consisted of 10-20 men and they were commanded by another slave. Fogel and Engerman address the "gang system" efficiency proposals in \citet{fogel1995time} and \citet{explaining}, arguing that the core of the "gang system" effect was driven by the size of farms, specially in the cotton plantations, and it yielded a fairly large efficiency in this field compared to its free workhand counterpart.

The second position argues that slavery did not have a positive effect on agricultural production. Wright claims that slavery did not existed just for the sake of being a medium of labor, but as a whole political and ideological system, that distinguished the "slaver" South from the "free" North in a deep level, and thus prevented the adoption of better ways of production. The slaves were used for crops in which farm labor productivity had to be enough to cover the price of the slave \citep[29--30]{wright2013slavery}, although in the North productivity in the crops was not high enough to support slaves.

According to Wright, the use of slavery was advantageous in the early times of American history. It allowed farmers to prepare the "yet to tame" conquered land in a short time, minimized the risk of loss in case of a poor harvest season and fully maximized the profits from a bountiful harvest. These characteristics shaped the agricultural sector of the South and made it lag behind eventually with the North.

Another relevant event, tied tightly to the phenomenon of slavery is the American Civil War (1861-1865). \cite{feigenbaum2018capital} make an analysis of the effect of the Sherman march on the land that suffered it. This study shows that the march deeply affected the production factors of the territories affected by big substantial changes in their allocations.
\section{Approach of the study}
This study will first start with the methodology used for researching this topic, beginning with the presentation of the data and ending with the empirical approach of the research. Next will follow the reasoned analysis of the results and their reliability. Lastly, the conclusion will deal with the discussion of the end results and their implications with the literature.

\chapter{Methodology}
\section{Data}
The data was obtained from the works of \citet{haines10} hosted in \citet{nhgis}, except for the information on agricultural prices in 1870, which was personally gathered from the \emph{Agriculture} section of \citet{histstat}.

The empirical work will use county-level agricultural and demographic data from the US Census of 1860 and 1870. This data includes variables such as livestock, non-crop agricultural production, production of several crops in particular, farm data and demographic data. This study focuses on typical crop productions that used mostly enslaved workhand, such as cotton or tobacco \footnote{According to \citet[29]{wright2013slavery}, tobacco employed two thirds of African slaves.}. 

This fact can be noticed visually by taking a look at the figures \ref{fig:cottoncomp} and \ref{fig:tobaccocomp} below. The scope of this study was expected be higher, unfortunately, there is a surprising decrease of detailed agricultural information in the 1870 Census. This means that some valuable information such as prices had to be obtained from other sources, as stated previously. Nevertheless, there is enough information to compute the effects on the total agricultural production and some sets of particular productions.

\begin{figure}[h!]
\centering
\includegraphics[width=1.1\textwidth]{cottoncomp} \caption{Comparison of cotton production per total production ratio with enslaved population ratio} \label{fig:cottoncomp}
\end{figure}

\begin{figure}[h!]
\centering
\includegraphics[width=1.1\textwidth]{tobaccocomp} \caption{Comparison of tobacco production per total production ratio with enslaved population ratio} \label{fig:tobaccocomp}
\end{figure}


For the demographic data, three variables were retrieved. Firstly the ratio of foreign-born population over total population. This is in order to estimate the effect of immigrant population on agricultural production, as opposed to the other great source of handwork: slavery. Some suspicions may arise from this data, as it does not make any distinction among population characteristics. However, \citet{wright1970economics} and \citet{russel1941effects} back this proceeding, arguing that women and children also participated in the labor of family farms. The workload may be unequal compared to their male, adult relatives, but certainly not enough to assume it was a "men only" job.

The second variable used is the population growth ratio. This was computed in order to estimate the effects of the American Civil War. Unfortunately, there is not enough data in county levels in order to make an exhaustive estimation of the damages of war, so this was the best choice available. The data was computed using male population older than 10 in 1860 and male population over 21 in 1870. Although the ages seem a bit young for war\footnote{Considering that the American Civil War started in 1861, young men recorded in the Census that enlisted in the Army would be 11 years old and older, although it is expected that the use of child soldiers would have been reserved by the last struggling moments of the war.}, there are some sources that demonstrate that the number of child recruits was actually very high \citep{murphy1993boys}, so it would not be unsound to add them in the estimations.

Other sources point out the role of women in the Civil War \citep{blanton2002they}, although at the time of computing the conscription of women the numbers are very low. The amount of women that joined the military "informally" could increase that value, yet the own uncertainty of that circumstance makes it very difficult to use them as a reliable estimator of the effects of war. Nevertheless, this might not be the best approach to estimate the damages of war, as \citet{feigenbaum2018capital} show that the demographic change might not be enough to estimate the lagging behind of Southern agriculture. Although the scope of their work is limited to the states of Georgia, North Carolina and South Carolina and might not be replicated to the rest of the Southern states or the South as a whole.

Lastly, the most important: slaves. The slave figures were retrieved for every type of slave registered in the census. This means that the data includes people of all sex and ages. This may also seem puzzling as with the data for foreign-born population, but \citet[73]{fogel1994without} also backs this approach, stating that "nearly one third of slave laborers were untrained children and about an eighth were elderly, crippled, or disadvantaged in some way. Women represented a much larger proportion of field laborers among slaves than among free farmers.". This is compelling evidence to include these other population sectors. Slave data is used in the regressions as the ratio of enslaved population per total population.

In order to keep the panel data consistent, many territories of 1870 that do not appear in the 1860 Census were dropped from the data. Also, the agricultural data measured in dollars in 1870 was standardized to 1860 levels using the information on the section \emph{Prices and Price Indexes} in \citet{histstat} \footnote{Farm products price index in 1860 was 77 and 112 in 1870, being 100 in 1910-1914 according to \citet{stine1933}.} to keep consistent and comparable values.

As the information is gathered in county levels, the number of observations in this database is sufficiently large to allow for precise econometric estimation. The nature of the data may be considered quasi-experimental, due to the fact that this change on the production of agriculture was caused by an institutional change, affecting a particular group of states with particular features. This characteristic of the data makes the differences in differences model very adequate for this study.

The differences in differences model will analyze the effect of slavery abolition in the Southern states, keeping as a counter-factual the agricultural production in the Northern states, which developed an agriculture free of slave workforce. The scope of this data may also contribute to estimate properly heterogeneous effects.

This data was constructed and analyzed using the program R \citep{R} and its many packages, mentioned in the bibliography. This work was written using \LaTeX \, \citep{latex2e}.
\section{Implementation}
In order to estimate the Differences in Differences model this work follows the indications of \citet{stock2003introduction}, \citet{wooldridge2007difference}, \citet{torres2010getting}, UC3M Applied Economics lectures and the articles of \citet{cohen2003effects} and \citet{felfe2015can}.

\subsection{OLS estimation}\label{sec:OLS estimation}

The following OLS equation will be used as a framework to estimate the effects of this policy change (slavery abolition) on agricultural output. All the classical assumptions are held and the standard errors are heteroskedasticity robust:
\begin{equation}
Y_{i,j,t}=\beta_{0}+\beta_{1}SLAVES_{i,j}+\beta_{2}AMEN_{i,j,t}+\beta_{3}(SLAVES_{i,j}\cdot AMEN_{i,j,t})+\delta_{i,j,t}X
\end{equation}
 where:
\begin{itemize}
\item $Y$ is the dependent variable. In the following studies, this variable will be linked to different agricultural outputs:
\begin{itemize}
\item $AGOUT3$ is the total agricultural output measured in dollars
\item $COTTONRAT$ is the ratio of cotton production per total agricultural output
\item $TOBACCORAT$ is the ratio of tobacco production per total agricultural output
\end{itemize}
\item $i$ is the county indicator.
\item $j$ is the state indicator, which takes a value of 1 if the observation takes place in a Confederate state, namely Alabama, Arkansas, Florida, Georgia, Kentucky, Louisiana, Mississippi, Missouri, North Carolina, South Carolina, Tennessee, Texas and Virginia.
\item $t$ is the time indicator.
\item $\beta_{0}$ is the constant.
\item $\beta_{1}$ is the estimator of the variable $SLAVES$, which is the percentage of enslaved population per total population.
\item $\beta_{2}$ is the estimator of the dummy variable $AMEN$, that takes a value of 1 if the observation belongs to the period after the introduction of the 13th Amendment, namely 1870.
\item $\beta_{3}$ is the estimator of the interaction term $SLAVES \cdot AMEN$ or diff. in diff. estimator. For the rest of the study this variable will be named as $DID$
\item $X$ is the set of explanatory variables:
\begin{itemize}
\item MACHINE is the value in dollars of machinery and implements.
\item LIVESTOCK is the value in dollars of livestock.
\item FARMS is the number of farms.
\item FORATIO is the ratio of foreign born people over total population.
\item POPGROWTH is the growth ratio of male population between the 1860 census and 1870 census.
\end{itemize}
\end{itemize}

In order to use this equation as a Cobb-Douglas production function that allows to interpret the estimators as elasticities, the previous general equation is transformed into this one:

\begin{equation}
ln(Y)_{i,j,t}=\beta_{0}+\beta_{1}SLAVES_{i,j}+\beta_{2}AMEN_{i,j,t}+\beta_{3}(SLAVES_{i,j}\cdot AMEN_{i,j,t})+\delta_{i,j,t}ln(X)
\end{equation}

where the variables are still the same as previously explained but through a logarithmic transformation, except for the ratios, which are not transformed.
\subsection{Panel under Fixed Effects model}

After the OLS model two other models were created including fixed and random effects estimators.

For these models, the base equation is the following:
\begin{equation}
Y_{i,j,t}=\beta_{1}X_{i,j,t} + a_{i,j}
\end{equation}

Here $Y_{i,j,t}$ and $X$ keep the same meaning as before, being $X$ the set of the different explanatory variables not constant over time plus the variables used to estimate the Diff-in-Diff. At last, $a_{i}$ is the unobserved effect. $a_{i,t}$ can also be considered $a_{i}=\beta_{0}+\beta_{2}Z_{i}$, where $\beta_{0}$ is the intersect and $\beta_{2}Z_{i}$ is the set of variables that stay constant over time while changing from one county to another \citep[349]{stock2003introduction}.

As in the OLS section, this framework equation is also transformed:
\begin{equation}
ln(Y)_{i,j,t}=\beta_{1}ln(X)_{i,j,t} + a_{i,j}
\end{equation}
In order to discern which model is more suitable, a Hausman Test is done comparing the fixed and random effects estimators. The results are shown in the table \ref{tab:phtest} below. They prove that there is enough evidence to safely reject the null hypothesis of the random effects model being insignificantly different from the fixed effects model. Then a fixed effects \footnote{From now on abbreviated as "FE"} model is chosen.

% latex table generated in R 3.6.0 by xtable 1.8-4 package
% Sat Jul 06 01:22:36 2019
\begin{table}[ht]
\centering
\caption{Hausman test for Fixed Effects vs Random Effects}
\label{tab:phtest}
\begin{tabular}{rrrrll}
  \hline
 & $\chi^{2}$ & p-value & parameter & method & alternative \\ 
  \hline
1 & 374.40 & < 2.2e-16 &   6 & Hausman Test & one model is inconsistent \\ 
   \hline
\end{tabular}
\end{table}

The estimations of both OLS and FE models are gathered in \nameref{sec:Appendix I}
\chapter{Results}
\section{Countrywide effects}
In the case of total agricultural production Table \ref{tab:agoutnopop} shows the result of the OLS: \begin{equation}\begin{split}
ln(AGOUT3)_{i,j,t} &=\beta_{0}+\beta_{1}SLAVES_{i,j}+\beta_{2}AMEN_{i,j,t}+\beta_{3}DID_{i,j,t}+\delta_{1}ln(MACHINE)_{i,j,t}+
\\
&\delta_{2}ln(LIVESTOCK)_{i,j,t}+\delta_{3}ln(FARMS)_{i,j,t}+\delta_{4}FORATIO_{i,j,t}
\end{split}
\end{equation}


and FE regression:\begin{equation}
\begin{split}
ln(AGOUT3)_{i,j,t} &=\beta_{2}AMEN_{i,j,t}+\beta_{3}DID_{i,j,t}+\delta_{1}ln(MACHINE)_{i,j,t}+
\\
&\delta_{2}ln(LIVESTOCK)_{i,j,t}+\delta_{3}ln(FARMS)_{i,j,t}+\delta_{4}FORATIO_{i,j,t} + a_{i,j}
\end{split}
\end{equation}


Here, the equations show the elasticities of the different production factors of agricultural production. The most interesting part comes from the DID estimator $\beta_{3}$ in Table \ref{tab:agoutnopop}, where it shows a positive elasticity in the OLS estimation while a negative elasticity in the FE model. This may seem contradicting, so in the \nameref{sec:Robustness check} section there are some indications that may help to discern which effect should be taken into account. In this case, and according to the conclusions reached in the aforementioned section, the effects of the FE are the main conductors of the study. This elasticity of $-0.592$ shows that for the policy change (slavery abolition), the aggregate agricultural output decreased around 60\%. Besides this, the estimator $\beta_{1}$ takes a value of $0.940$, meaning that for an increase of 1\% in the slave ratio the agricultural output will increase by 0.90\%.

For the case of cotton production per total the equations are the following:
\begin{equation}\begin{split}
ln(COTTONRAT)_{i,j,t} &=\beta_{0}+\beta_{1}SLAVES_{i,j}+\beta_{2}AMEN_{i,j,t}+\beta_{3}DID_{i,j,t}+\delta_{1}ln(MACHINE)_{i,j,t}
\\
&+\delta_{2}ln(LIVESTOCK)_{i,j,t}+\delta_{3}ln(FARMS)_{i,j,t}+\delta_{4}FORATIO_{i,j,t}
\end{split}
\end{equation}

\begin{equation}
\begin{split}
ln(COTTONRAT)_{i,j,t} &=\beta_{2}AMEN_{i,j,t}+\beta_{3}DID_{i,j,t}+\delta_{1}ln(MACHINE)_{i,j,t}+
\\
&\delta_{2}ln(LIVESTOCK)_{i,j,t}+\delta_{3}ln(FARMS)_{i,j,t}+\delta_{4}FORATIO_{i,j,t} + a_{i,j}
\end{split}
\end{equation}

whose estimations are gathered in Table \ref{tab:cotnopop}. Here $\beta_{3}$ takes similar values in both OLS and FE models, so it is easy to say that the abolition of slavery reduced the ratio of cotton production per total agricultural production by 22\%. In the case of the SLAVES estimator, the result is $\beta_{1}=0.756$, meaning that for an increase of 1\% in the slave ratio the cotton ratio would increase by a 0.75\%.

Similarly, for the tobacco ratio the models are the following:
\begin{equation}\begin{split}
ln(TOBACCORAT)_{i,j,t} &=\beta_{0}+\beta_{1}SLAVES_{i,j}+\beta_{2}AMEN_{i,j,t}+\beta_{3}DID_{i,j,t}+\delta_{1}ln(MACHINE)_{i,j,t}
\\
&+\delta_{2}ln(LIVESTOCK)_{i,j,t}+\delta_{3}ln(FARMS)_{i,j,t}+\delta_{4}FORATIO_{i,j,t}
\end{split}
\end{equation}

\begin{equation}
\begin{split}
ln(TOBACCORAT)_{i,j,t} &=\beta_{2}AMEN_{i,j,t}+\beta_{3}DID_{i,j,t}+\delta_{1}ln(MACHINE)_{i,j,t}+
\\
&\delta_{2}ln(LIVESTOCK)_{i,j,t}+\delta_{3}ln(FARMS)_{i,j,t}+\delta_{4}FORATIO_{i,j,t} + a_{i,j}
\end{split}
\end{equation}

whose estimations are also gathered in Table \ref{tab:tobnopop}. In this case the outcomes are more dissimilar than in the cotton case, being the FE estimator more "conservative" with the effects of abolition. It estimates an elasticity of $\beta_{3}=-0.023$ or that for a change in the policy the ratio of tobacco output per total agricultural output decreases by a 2.3\%. This is by far the lowest negative effect of slavery abolition on any agricultural product analyzed here. Also, the role of the estimator $\beta_{1}$ is very small, taking just a value of $0.085$, meaning that for an increase of 1\% in the slave ratio the tobacco ratio would only increase by 0.085\%. This goes in line with \citet[29]{wright2013slavery}, that argues that the growth of tobacco was not the most suited for the slave kind of work.

\section{Heterogeneous treatment effects}
There is a sizable amount of literature that deal with the analysis of heterogeneous treatment effects on the different groups using quasi-experimental data. Two articles that address this concept and were used to guide this part of the study were \cite{felfe2015can} and \cite{NBERw9708}. The group that this work will be focused on is the Southern or Confederate States. These states were already defined in the \nameref{sec:OLS estimation} section, so the equations used for the OLS and FE estimations will be the following:
\begin{equation}
\begin{split}
ln(AGOUT3)_{i,1,t} &=\beta_{0}+\beta_{1}SLAVES_{i,1}+\beta_{2}AMEN_{i,1,t}+\beta_{3}DID_{i,1,t}
\\
&+\delta_{1}MACHINE_{i,1,t}+\delta_{2}LIVESTOCK_{i,1,t}+\delta_{3}FARMS_{i,1,t}+\delta_{4}FORATIO_{i,1,t}
\end{split}
\end{equation}

\begin{equation}
\begin{split}
ln(AGOUT3)_{i,1,t} &=\beta_{2}AMEN_{i,1,t}+\beta_{3}DID_{i,1,t}+\delta_{1}ln(MACHINE)_{i,1,t}+
\\
&\delta_{2}ln(LIVESTOCK)_{i,1,t}+\delta_{3}ln(FARMS)_{i,1,t}+\delta_{4}FORATIO_{i,1,t} + a_{i,1}
\end{split}
\end{equation}


The results of this regression, which are gathered in Table \ref{tab:agoutnopop}, show a similar trend as with the whole country. Once again the estimates of $\beta_{3}$ are different from the OLS to the FE model and once again the estimator from the FE model is used as a reference. Here, it takes a value of -0.530, meaning that with the abolition of slavery the Confederate states lost more than 50\% of the total agricultural output. Slave ratio also has a higher effect than in the countrywide production, with a estimation of $\beta_{1}=1.611$, showing that with an increase of 1\% in the slave ratio the aggregate output would hike by a 1.611\%.

Now, for the cotton production, the equations are very similar as before:

\begin{equation}
\begin{split}
ln(COTTONRAT)_{i,1,t} &=\beta_{0}+\beta_{1}SLAVES_{i,1}+\beta_{2}AMEN_{i,1,t}+\beta_{3}DID_{i,1,t}
\\
&+\delta_{1}MACHINE_{i,1,t}+\delta_{2}LIVESTOCK_{i,1,t}+\delta_{3}FARMS_{i,1,t}+\delta_{4}FORATIO_{i,1,t}
\end{split}
\end{equation}

\begin{equation}
\begin{split}
ln(COTTONRAT)_{i,1,t} &=\beta_{2}AMEN_{i,1,t}+\beta_{3}DID_{i,1,t}+\delta_{1}ln(MACHINE)_{i,1,t}+
\\
&\delta_{2}ln(LIVESTOCK)_{i,1,t}+\delta_{3}ln(FARMS)_{i,1,t}+\delta_{4}FORATIO_{i,1,t} + a_{i,1}
\end{split}
\end{equation}

and having the same problem with the estimators in OLS and FE as before. Here the rule of thumb is also followed, so according to table \ref{tab:cotnopopcsa}, $\beta_{3}=-0.107$. This means that with the slavery abolition policy, the ratio of cotton production per total agricultural production decreased by a 10\%. The effect of slavery also increases a small amount compared to the countrywide production. In this case $\beta_{1}=0.810$ shows that the increase of the slave ratio by 1\% would also increase the cotton ratio by 0.81\%.

At last, the analysis for tobacco production come from these models, also very similar as previous steps:
\begin{equation}
\begin{split}
ln(TOBACCORAT)_{i,1,t} &=\beta_{0}+\beta_{1}SLAVES_{i,1}+\beta_{2}AMEN_{i,1,t}+\beta_{3}DID_{i,1,t}
\\
&+\delta_{1}MACHINE_{i,1,t}+\delta_{2}LIVESTOCK_{i,1,t}+\delta_{3}FARMS_{i,1,t}+\delta_{4}FORATIO_{i,1,t}
\end{split}
\end{equation}

\begin{equation}
\begin{split}
ln(TOBACCORAT)_{i,1,t} &=\beta_{2}AMEN_{i,1,t}+\beta_{3}DID_{i,1,t}+\delta_{1}ln(MACHINE)_{i,1,t}+
\\
&\delta_{2}ln(LIVESTOCK)_{i,1,t}+\delta_{3}ln(FARMS)_{i,1,t}+\delta_{4}FORATIO_{i,1,t} + a_{i,1}
\end{split}
\end{equation}

Here the results of the estimators also vary from the OLS model to the FE one and once again the FE estimator is also preferred. The $\beta_{3}$ estimator in table \ref{tab:tobnopopcsa} takes a value of -0.530, which means that after the abolition of slavery the ratio of tobacco production per total production plummeted by more than a 50\%. For the slave effect, it seems that  is even smaller than in the countrywide environment, with an elasticity of 0.044. This only reinforces the statement of \citet{wright2013slavery} on tobacco production.

\section{Robustness check}\label{sec:Robustness check}
At first, the main concern that may arise from an analysis in the 1860-70 period in the USA is, of course, the effects of the American Civil War. In order to see whether the estimations are affected by any kind of bias, the base OLS regression now includes the variable of population growth:

\begin{equation}\begin{split}
Y_{i,j,t} &=\beta_{0}+\beta_{1}SLAVES_{i,j}+\beta_{2}AMEN_{i,j,t}+\beta_{3}DID_{i,j,t}+\delta_{1}ln(MACHINE)_{i,j,t}+
\\
&\delta_{2}ln(LIVESTOCK)_{i,j,t}+\delta_{3}ln(FARMS)_{i,j,t}+\delta_{4}FORATIO_{i,j,t}
\\
&+\delta_{5}POPGROWTH_{i,j}
\end{split}
\end{equation}

And according to tables \ref{tab:agout}, \ref{tab:cot}, \ref{tab:tob}, \ref{tab:agoutcsa} and \ref{tab:cotcsa} the estimator $\delta_{5}$ is not significant in any of the instances, except for the tobacco ratio in the Southern states (Table \ref{tab:tobcsa}), in which the estimator is only significant at 10\% but the coefficient is almost insignificant.

A visual comparison of different maps, one displaying agricultural outputs and other displaying the population change, corroborates that intuition of population change not being relevant to agricultural production:


\begin{figure}[h!]
\centering
\includegraphics[width=1.1\textwidth]{cottonpopcomp} \caption{Comparison of cotton production with population growth} \label{fig:cottonpopcomp}
\end{figure}

\begin{figure}[h!]
\centering
\includegraphics[width=1.1\textwidth]{agoutpopcomp} \caption{Comparison of total agricultural production with population growth} \label{fig:agoutpopcomp}
\end{figure}

\begin{figure}[h!]
\centering
\includegraphics[width=1.1\textwidth]{tobaccopopsouthcomp} \caption{Comparison of tobacco production with population growth in the Confederate States} \label{fig:tobaccopopsouthcomp}
\end{figure}

Here in figure \ref{fig:tobaccopopsouthcomp} it can be seen a small correlation in some parts of West Florida. However, for the main tobacco grower regions of Kentucky and Virginia this hypothesis is not maintained.

This may prove right the hypothesis of population change not being one of the main drivers of the effect of war on agricultural output of \citet{feigenbaum2018capital}. In fact, war horrors are tragic from a humanitarian standpoint, but from an economic perspective, they are just large shocks on the national economy from which it eventually recovers. This is the case of postwar Europe, that in 1947 already caught up prewar productive capital levels, according to \citet[52]{eichengreen2008european}. Feigenbaum et al. point out other factors as the main drivers of the war shocks, such as an undeveloped financial sector. Unfortunately, there is not sufficient data to estimate the effects of the financial sector in the after war period.

Another question arises, which is what model should be chosen. The literature says that FE is preferred to a pooled OLS model, but in any case this work includes empirical evidence to corroborate the general consensus. In this case an F-test for individual effects was performed. The null hypothesis states that there are not significant effects, while the alternative hypothesis states that there are indeed significant effects. The test results are gathered in the table \ref{tab:pFtest} below:
% latex table generated in R 3.6.0 by xtable 1.8-4 package
% Sat Jul 06 01:28:30 2019
\begin{table}[ht]
\centering
\caption{F-test for Pooled OLS vs Fixed Effects}
\label{tab:pFtest}
\begin{tabular}{rrrrrll}
  \hline
 & df1 & df2 & F-statistic & p-value & method & alternative \\ 
  \hline
1 & 1905 & 1825 & 2.31 & < 2.2e-16 & F test for individual effects & significant effects \\ 
   \hline
\end{tabular}
\end{table}

It shows that with a p-value lower than 1\% the null hypothesis is rejected, so in case of doubt, the FE model can be considered less biased than the OLS and its conclusions are preferred.



\chapter{Conclusion}
Now that the results are shown and explained it is time to discuss them and explain their implications. These results are the empirical evidence that slavery did have a positive effect on agricultural production and the abolition of slavery meant a negative effect on the agricultural production. In fact, these results corroborate overwhelmingly the proposals of Fogle and Engerman. They show the big importance of enslaved labor on the Cobb-Douglas production function, with considerably high elasticities for slaves. This means that their hypothesis of enslaved labor was right and this kind of labor was efficient and more than with free workhand. 

The DID model also shows that once the 13th Amendment was passed, this efficient system was eliminated and that meant a decrease and lag behind of the Southern agriculture, which is the sector of the US that relied the most on slavery. 
This does not mean that the position of Wright is wrong. In fact, many results have proved right many of his points, such as the reduced efficiency of slavery in tobacco production. Furthermore, these results only cover the 1860-1870 period, and probably for a wider period they could change and give a different interpretation for the abolition of slavery.

Another issue that can support the position of Wright is the possible underestimation of war effects. Although it is true that in the \nameref{sec:Robustness check} section the effect of population growth was not significant, it is likely that there are more significant ways to measure the impacts of war besides the war casualties. Nevertheless, the data available for such an early period in history is quite limited, and so are the possibilities of replicating historical environments with more accuracy and using more tools to support the results.

To conclude this discussion, although the evidence here points out that slavery in the 19th Century was a more efficient system than the free worker system, it is worth mentioning that this fact does not justify or condone slavery. In fact, this drives away many weak arguments of economic profitability and shifts the debate to a more important matter: individual freedom, a defining characteristic of the United States of America.
%----------
%	BIBLIOGRAFÍA
%----------	

\nocite{*} % Si quieres que aparezcan en la bibliografía todos los documentos que la componen (también los que no estén citados en el texto) descomenta está lína

\clearpage
\addcontentsline{toc}{chapter}{Bibliography}

\printbibliography

%----------
%	ANEXOS
%----------	

% Si tu trabajo incluye anexos, puedes descomentar las siguientes líneas

\chapter*{Appendix I}\label{sec:Appendix I}
\pagenumbering{gobble} % Las páginas de los anexos no se numeran
\section*{Estimations with no population growth}


% Table created by stargazer v.5.2.2 by Marek Hlavac, Harvard University. E-mail: hlavac at fas.harvard.edu
% Date and time: ju., jul. 04, 2019 - 18:59:52
\begin{table}[!htbp] \centering 
  \caption{Total agricultural production} 
  \label{tab:agoutnopop} 
\begin{tabular}{@{\extracolsep{5pt}}lcc} 
\\[-1.8ex]\hline 
\hline \\[-1.8ex] 
 & \multicolumn{2}{c}{\textit{Dependent variable:}} \\ 
\cline{2-3} 
\\[-1.8ex] & \multicolumn{2}{c}{Aggregate agricultural output (logs)} \\ 
\\[-1.8ex] & \textit{OLS} & \textit{FE} \\ 
\\[-1.8ex] & (1) & (2)\\ 
\hline \\[-1.8ex] 
 Slave ratio & 0.940$^{***}$ &  \\ 
  & p = 0.000 &  \\ 
  Before the 13th Amendment & 0.00001 & 0.075$^{***}$ \\ 
  & p = 1.000 & p = 0.00000 \\ 
  Diff-in-diff estimator & 0.279$^{***}$ & $-$0.592$^{***}$ \\ 
  & p = 0.00000 & p = 0.000 \\ 
  Implements and machinery (logs) & 0.374$^{***}$ & 0.186$^{***}$ \\ 
  & p = 0.000 & p = 0.000 \\ 
  Livestock (logs) & 0.367$^{***}$ & 0.218$^{***}$ \\ 
  & p = 0.000 & p = 0.000 \\ 
  Number of farms (logs) & 0.287$^{***}$ & 0.421$^{***}$ \\ 
  & p = 0.000 & p = 0.000 \\ 
  Foreign people ratio & 0.089 & $-$0.696$^{**}$ \\ 
  & p = 0.121 & p = 0.015 \\ 
  Constant & 2.261$^{***}$ &  \\ 
  & p = 0.000 &  \\ 
 \hline \\[-1.8ex] 
Observations & 3,738 & 3,738 \\ 
R$^{2}$ & 0.955 & 0.699 \\ 
Adjusted R$^{2}$ & 0.955 & 0.384 \\ 
F Statistic &  & 707.601$^{***}$ (df = 6; 1825) \\ 
\hline 
\hline \\[-1.8ex] 
\textit{Note:}  & \multicolumn{2}{r}{$^{*}$p$<$0.1; $^{**}$p$<$0.05; $^{***}$p$<$0.01} \\ 
\end{tabular} 
\end{table} 

% Table created by stargazer v.5.2.2 by Marek Hlavac, Harvard University. E-mail: hlavac at fas.harvard.edu
% Date and time: vi., jul. 05, 2019 - 12:58:38
\begin{table}[!htbp] \centering 
  \caption{Cotton production per total agricultural production} 
  \label{tab:cotnopop} 
\begin{tabular}{@{\extracolsep{5pt}}lcc} 
\\[-1.8ex]\hline 
\hline \\[-1.8ex] 
 & \multicolumn{2}{c}{\textit{Dependent variable:}} \\ 
\cline{2-3} 
\\[-1.8ex] & \multicolumn{2}{c}{Cotton ratio} \\ 
\\[-1.8ex] & \textit{OLS} & \textit{FE} \\  
\\[-1.8ex] & (1) & (2)\\ 
\hline \\[-1.8ex] 
 Slave ratio & 0.756$^{***}$ &  \\ 
  & p = 0.000 &  \\ 
  Before the 13th Amendment & 0.002 & 0.002 \\ 
  & p = 0.819 & p = 0.734 \\ 
  Diff-in-diff estimator & $-$0.226$^{***}$ & $-$0.229$^{***}$ \\ 
  & p = 0.000 & p = 0.000 \\ 
  Implements and machinery (logs) & $-$0.033$^{***}$ & $-$0.008 \\ 
  & p = 0.000 & p = 0.271 \\ 
  Livestock (logs) & 0.029$^{***}$ & $-$0.003 \\ 
  & p = 0.00000 & p = 0.770 \\ 
  Number of farms (logs) & 0.007 & 0.016$^{*}$ \\ 
  & p = 0.255 & p = 0.097 \\ 
  Foreign people ratio & 0.014 & 0.005 \\ 
  & p = 0.588 & p = 0.964 \\ 
  Constant & $-$0.050 &  \\ 
  & p = 0.176 &  \\ 
 \hline \\[-1.8ex] 
Observations & 3,738 & 3,738 \\ 
R$^{2}$ & 0.409 & 0.085 \\ 
Adjusted R$^{2}$ & 0.407 & $-$0.875 \\ 
F Statistic & 368.020$^{***}$ (df = 7; 3730) & 28.094$^{***}$ (df = 6; 1825) \\ 
\hline 
\hline \\[-1.8ex] 
\textit{Note:}  & \multicolumn{2}{r}{$^{*}$p$<$0.1; $^{**}$p$<$0.05; $^{***}$p$<$0.01} \\ 
\end{tabular} 
\end{table} 


% Table created by stargazer v.5.2.2 by Marek Hlavac, Harvard University. E-mail: hlavac at fas.harvard.edu
% Date and time: vi., jul. 05, 2019 - 12:57:31
\begin{table}[!htbp] \centering 
  \caption{Tobacco production per total agricultural production} 
  \label{tab:tobnopop} 
\begin{tabular}{@{\extracolsep{5pt}}lcc} 
\\[-1.8ex]\hline 
\hline \\[-1.8ex] 
 & \multicolumn{2}{c}{\textit{Dependent variable:}} \\ 
\cline{2-3} 
\\[-1.8ex] & \multicolumn{2}{c}{Tobacco ratio} \\ 
\\[-1.8ex] & \textit{OLS} & \textit{FE} \\ 
\\[-1.8ex] & (1) & (2)\\ 
\hline \\[-1.8ex] 
 Slave ratio & 0.085$^{***}$ &  \\ 
  & p = 0.000 &  \\ 
  Before the 13th Amendment & $-$0.003 & 0.002 \\ 
  & p = 0.287 & p = 0.206 \\ 
  Diff-in-diff estimator & $-$0.050$^{***}$ & $-$0.023$^{***}$ \\ 
  & p = 0.00003 & p = 0.00003 \\ 
  Implements and machinery & $-$0.005$^{**}$ & $-$0.001 \\ 
  & p = 0.012 & p = 0.642 \\ 
  Livestock & $-$0.008$^{***}$ & 0.0004 \\ 
  & p = 0.001 & p = 0.832 \\ 
  Number of farms & 0.019$^{***}$ & $-$0.0003 \\ 
  & p = 0.000 & p = 0.881 \\ 
  Foreign people ratio & $-$0.039$^{***}$ & $-$0.013 \\ 
  & p = 0.0005 & p = 0.614 \\ 
  Constant & 0.055$^{***}$ &  \\ 
  & p = 0.0005 &  \\ 
 \hline \\[-1.8ex] 
Observations & 3,738 & 3,738 \\ 
R$^{2}$ & 0.053 & 0.021 \\ 
Adjusted R$^{2}$ & 0.051 & $-$1.005 \\ 
F Statistic & 29.618$^{***}$ (df = 7; 3730) & 6.433$^{***}$ (df = 6; 1825) \\ 
\hline 
\hline \\[-1.8ex] 
\textit{Note:}  & \multicolumn{2}{r}{$^{*}$p$<$0.1; $^{**}$p$<$0.05; $^{***}$p$<$0.01} \\ 
\end{tabular} 
\end{table}  

\newpage

\subsection*{Heterogeneity}


% Table created by stargazer v.5.2.2 by Marek Hlavac, Harvard University. E-mail: hlavac at fas.harvard.edu
% Date and time: ju., jul. 04, 2019 - 19:03:00
\begin{table}[!htbp] \centering 
  \caption{Total agricultural production in confederate states} 
  \label{tab:agoutnopopcsa} 
\begin{tabular}{@{\extracolsep{5pt}}lcc} 
\\[-1.8ex]\hline 
\hline \\[-1.8ex] 
 & \multicolumn{2}{c}{\textit{Dependent variable:}} \\ 
\cline{2-3} 
\\[-1.8ex] & \multicolumn{2}{c}{Aggregate agricultural output (logs)} \\ 
\\[-1.8ex] & \textit{OLS} & \textit{FE} \\ 

\\[-1.8ex] & (1) & (2)\\ 
\hline \\[-1.8ex] 
 Slave ratio & 1.611$^{***}$ &  \\ 
  & p = 0.000 &  \\ 
  Before the 13th Amendment & 0.097$^{***}$ & 0.066$^{***}$ \\ 
  & p = 0.003 & p = 0.0002 \\ 
  Diff-in-diff estimator & $-$0.324$^{***}$ & $-$0.530$^{***}$ \\ 
  & p = 0.0002 & p = 0.000 \\ 
  Implements and machinery (logs) & 0.354$^{***}$ & 0.147$^{***}$ \\ 
  & p = 0.000 & p = 0.000 \\ 
  Livestock (logs) & 0.195$^{***}$ & 0.336$^{***}$ \\ 
  & p = 0.000 & p = 0.000 \\ 
  Number of farms (logs) & 0.454$^{***}$ & 0.332$^{***}$ \\ 
  & p = 0.000 & p = 0.000 \\ 
  Foreign people ratio & $-$0.581$^{***}$ & 0.521 \\ 
  & p = 0.00003 & p = 0.144 \\ 
  Constant & 3.381$^{***}$ &  \\ 
  & p = 0.000 &  \\ 
 \hline \\[-1.8ex] 
Observations & 2,035 & 2,035 \\ 
R$^{2}$ & 0.869 & 0.749 \\ 
Adjusted R$^{2}$ & 0.869 & 0.478 \\ 
F Statistic & 1,926.934$^{***}$ (df = 7; 2027) & 486.082$^{***}$ (df = 6; 977) \\ 
\hline 
\hline \\[-1.8ex] 
\textit{Note:}  & \multicolumn{2}{r}{$^{*}$p$<$0.1; $^{**}$p$<$0.05; $^{***}$p$<$0.01} \\ 
\end{tabular} 
\end{table} 


% Table created by stargazer v.5.2.2 by Marek Hlavac, Harvard University. E-mail: hlavac at fas.harvard.edu
% Date and time: ju., jul. 04, 2019 - 19:43:47
\begin{table}[!htbp] \centering 
  \caption{Cotton production per total agricultural production in confederate states} 
  \label{tab:cotnopopcsa} 
\begin{tabular}{@{\extracolsep{5pt}}lcc} 
\\[-1.8ex]\hline 
\hline \\[-1.8ex] 
 & \multicolumn{2}{c}{\textit{Dependent variable:}} \\ 
\cline{2-3} 
\\[-1.8ex] & \multicolumn{2}{c}{Cotton ratio} \\ 
\\[-1.8ex] & \textit{OLS} & \textit{FE} \\ 
\\[-1.8ex] & (1) & (2)\\ 
\hline \\[-1.8ex] 
 Slave ratio & 0.810$^{***}$ &  \\ 
  & p = 0.000 &  \\ 
  Before the 13th Amendment & 0.010 & $-$0.001 \\ 
  & p = 0.601 & p = 0.790 \\ 
  Diff-in-diff estimator & $-$0.252$^{***}$ & $-$0.107$^{***}$ \\ 
  & p = 0.00000 & p = 0.00000 \\ 
  Implements and machinery (logs) & $-$0.052$^{***}$ & 0.029$^{***}$ \\ 
  & p = 0.000 & p = 0.00000 \\ 
  Livestock (logs) & 0.050$^{***}$ & $-$0.0005 \\ 
  & p = 0.00001 & p = 0.957 \\ 
  Number of farms (logs) & 0.003 & $-$0.024$^{***}$ \\ 
  & p = 0.740 & p = 0.002 \\ 
  Foreign people ratio & $-$0.072 & $-$0.049 \\ 
  & p = 0.352 & p = 0.586 \\ 
  Constant & $-$0.102 &  \\ 
  & p = 0.223 &  \\ 
 \hline \\[-1.8ex] 
Observations & 2,035 & 2,035 \\ 
R$^{2}$ & 0.283 & 0.279 \\ 
Adjusted R$^{2}$ & 0.280 & $-$0.500 \\ 
F Statistic & 114.206$^{***}$ (df = 7; 2027) & 63.136$^{***}$ (df = 6; 977) \\ 
\hline 
\hline \\[-1.8ex] 
\textit{Note:}  & \multicolumn{2}{r}{$^{*}$p$<$0.1; $^{**}$p$<$0.05; $^{***}$p$<$0.01} \\ 
\end{tabular} 
\end{table} 


% Table created by stargazer v.5.2.2 by Marek Hlavac, Harvard University. E-mail: hlavac at fas.harvard.edu
% Date and time: ju., jul. 04, 2019 - 19:03:02
\begin{table}[!htbp] \centering 
  \caption{Tobacco production per total agricultural production in confederate states} 
  \label{tab:tobnopopcsa} 
\begin{tabular}{@{\extracolsep{5pt}}lcc} 
\\[-1.8ex]\hline 
\hline \\[-1.8ex] 
 & \multicolumn{2}{c}{\textit{Dependent variable:}} \\ 
\cline{2-3} 
\\[-1.8ex] & \multicolumn{2}{c}{Tobacco ratio} \\ 
\\[-1.8ex] & \textit{OLS} & \textit{FE} \\ 
\\[-1.8ex] & (1) & (2)\\ 
\hline \\[-1.8ex] 
 Slave ratio & 0.044$^{***}$ &  \\ 
  & p = 0.005 &  \\ 
  Before the 13th Amendment & $-$0.005 & 0.002 \\ 
  & p = 0.514 & p = 0.362 \\ 
  Diff-in-diff estimator & $-$0.046$^{**}$ & $-$0.026$^{***}$ \\ 
  & p = 0.018 & p = 0.001 \\ 
  Implements and machinery (logs) & 0.003 & $-$0.003 \\ 
  & p = 0.452 & p = 0.218 \\ 
  Livestock (logs) & $-$0.013$^{***}$ & 0.004 \\ 
  & p = 0.004 & p = 0.285 \\ 
  Number of farms (logs) & 0.026$^{***}$ & $-$0.002 \\ 
  & p = 0.000 & p = 0.591 \\ 
  Foreign people ratio & 0.012 & $-$0.017 \\ 
  & p = 0.695 & p = 0.622 \\ 
  Constant & $-$0.001 &  \\ 
  & p = 0.975 &  \\ 
 \hline \\[-1.8ex] 
Observations & 2,035 & 2,035 \\ 
R$^{2}$ & 0.040 & 0.033 \\ 
Adjusted R$^{2}$ & 0.037 & $-$1.013 \\ 
F Statistic & 12.203$^{***}$ (df = 7; 2027) & 5.541$^{***}$ (df = 6; 977) \\ 
\hline 
\hline \\[-1.8ex] 
\textit{Note:}  & \multicolumn{2}{r}{$^{*}$p$<$0.1; $^{**}$p$<$0.05; $^{***}$p$<$0.01} \\ 
\end{tabular} 
\end{table} 


\newpage

\section*{Estimations with population growth}


% Table created by stargazer v.5.2.2 by Marek Hlavac, Harvard University. E-mail: hlavac at fas.harvard.edu
% Date and time: ju., jul. 04, 2019 - 19:05:44
\begin{table}[!htbp] \centering 
  \caption{Total agricultural production including population growth} 
  \label{tab:agout} 
\begin{tabular}{@{\extracolsep{5pt}}lcc} 
\\[-1.8ex]\hline 
\hline \\[-1.8ex] 
 & \multicolumn{2}{c}{\textit{Dependent variable:}} \\ 
\cline{2-3} 
\\[-1.8ex] & \multicolumn{2}{c}{Agregate agricultural output (logs)} \\ 
\\[-1.8ex] & \textit{OLS} & \textit{FE} \\ 
\\[-1.8ex] & (1) & (2)\\ 
\hline \\[-1.8ex] 
 Slave ratio & 0.931$^{***}$ &  \\ 
  & p = 0.000 &  \\ 
  Before the 13th Amendment & 0.001 & 0.075$^{***}$ \\ 
  & p = 0.916 & p = 0.00000 \\ 
  Diff-in-diff estimator & 0.277$^{***}$ & $-$0.592$^{***}$ \\ 
  & p = 0.00000 & p = 0.000 \\ 
  Implements and machinery (logs) & 0.376$^{***}$ & 0.186$^{***}$ \\ 
  & p = 0.000 & p = 0.000 \\ 
  Livestock (logs) & 0.364$^{***}$ & 0.218$^{***}$ \\ 
  & p = 0.000 & p = 0.000 \\ 
  Number of farms (logs) & 0.285$^{***}$ & 0.421$^{***}$ \\ 
  & p = 0.000 & p = 0.000 \\ 
  Foreign people ratio & 0.108$^{*}$ & $-$0.696$^{**}$ \\ 
  & p = 0.075 & p = 0.015 \\ 
  Population growth rate 1860-1870 & $-$0.015 &  \\ 
  & p = 0.333 &  \\ 
  Constant & 2.285$^{***}$ &  \\ 
  & p = 0.000 &  \\ 
 \hline \\[-1.8ex] 
Observations & 3,738 & 3,738 \\ 
R$^{2}$ & 0.955 & 0.699 \\ 
Adjusted R$^{2}$ & 0.955 & 0.384 \\ 
F Statistic &  & 707.601$^{***}$ (df = 6; 1825) \\ 
\hline 
\hline \\[-1.8ex] 
\textit{Note:}  & \multicolumn{2}{r}{$^{*}$p$<$0.1; $^{**}$p$<$0.05; $^{***}$p$<$0.01} \\ 
\end{tabular} 
\end{table} 


% Table created by stargazer v.5.2.2 by Marek Hlavac, Harvard University. E-mail: hlavac at fas.harvard.edu
% Date and time: ma., jul. 02, 2019 - 18:23:45
\begin{table}[!htbp] \centering 
  \caption{Cotton production per total agricultural production including population growth} 
  \label{tab:cot} 
\begin{tabular}{@{\extracolsep{5pt}}lcc} 
\\[-1.8ex]\hline 
\hline \\[-1.8ex] 
 & \multicolumn{2}{c}{\textit{Dependent variable:}} \\ 
\cline{2-3} 
\\[-1.8ex] & \multicolumn{2}{c}{Cotton ratio} \\ 
\\[-1.8ex] & \textit{OLS} & \textit{FE} \\ 
\\[-1.8ex] & (1) & (2)\\ 
\hline \\[-1.8ex] 
 Slave ratio & 0.743$^{***}$ & 0.740$^{***}$ \\ 
  & p = 0.000 & p = 0.000 \\ 
  Before the 13th Amendment & 0.008 & 0.006 \\ 
  & p = 0.294 & p = 0.427 \\ 
  Diff-in-diff estimator & $-$0.219$^{***}$ & $-$0.206$^{***}$ \\ 
  & p = 0.000 & p = 0.000 \\ 
  Implements and machinery (logs) & $-$0.00000$^{***}$ &  \\ 
  & p = 0.0003 &  \\ 
  Livestock (logs) & 0.00000$^{*}$ &  \\ 
  & p = 0.052 &  \\ 
  Number of farms (logs) & 0.00000 &  \\ 
  & p = 0.946 &  \\ 
  Foreign people ratio & $-$0.006 &  \\ 
  & p = 0.832 &  \\ 
  Population growth rate 1860-1870 & $-$0.001 &  \\ 
  & p = 0.868 &  \\ 
  Constant & 0.001 & $-$0.001 \\ 
  & p = 0.850 & p = 0.893 \\ 
 \hline \\[-1.8ex] 
Observations & 3,750 & 3,750 \\ 
R$^{2}$ & 0.403 & 0.400 \\ 
Adjusted R$^{2}$ & 0.402 & 0.400 \\ 
F Statistic & 315.388$^{***}$ (df = 8; 3741) & 832.846$^{***}$ (df = 3; 3746) \\ 
\hline 
\hline \\[-1.8ex] 
\textit{Note:}  & \multicolumn{2}{r}{$^{*}$p$<$0.1; $^{**}$p$<$0.05; $^{***}$p$<$0.01} \\ 
\end{tabular} 
\end{table} 

% Table created by stargazer v.5.2.2 by Marek Hlavac, Harvard University. E-mail: hlavac at fas.harvard.edu
% Date and time: ma., jul. 02, 2019 - 18:23:46
\begin{table}[!htbp] \centering 
  \caption{Tobacco production per total agricultural production including population growth} 
  \label{tab:tob} 
\begin{tabular}{@{\extracolsep{5pt}}lcc} 
\\[-1.8ex]\hline 
\hline \\[-1.8ex] 
 & \multicolumn{2}{c}{\textit{Dependent variable:}} \\ 
\cline{2-3} 
\\[-1.8ex] & \multicolumn{2}{c}{Tobacco ratio} \\ 
\\[-1.8ex] & \textit{OLS} & \textit{FE} \\ 
\\[-1.8ex] & (1) & (2)\\ 
\hline \\[-1.8ex] 
 Slave ratio & 0.743$^{***}$ & 0.740$^{***}$ \\ 
  & p = 0.000 & p = 0.000 \\ 
  Before the 13th Amendment & 0.008 & 0.006 \\ 
  & p = 0.294 & p = 0.427 \\ 
  Diff-in-diff estimator & $-$0.219$^{***}$ & $-$0.206$^{***}$ \\ 
  & p = 0.000 & p = 0.000 \\ 
  Implements and machinery (logs) & $-$0.00000$^{***}$ &  \\ 
  & p = 0.0003 &  \\ 
  Livestock (logs) & 0.00000$^{*}$ &  \\ 
  & p = 0.052 &  \\ 
  Number of farms (logs) & 0.00000 &  \\ 
  & p = 0.946 &  \\ 
  Foreign people ratio & $-$0.006 &  \\ 
  & p = 0.832 &  \\ 
  Population growth rate 1860-1870 & $-$0.001 &  \\ 
  & p = 0.868 &  \\ 
  Constant & 0.001 & $-$0.001 \\ 
  & p = 0.850 & p = 0.893 \\ 
 \hline \\[-1.8ex] 
Observations & 3,750 & 3,750 \\ 
R$^{2}$ & 0.403 & 0.400 \\ 
Adjusted R$^{2}$ & 0.402 & 0.400 \\ 
F Statistic & 315.388$^{***}$ (df = 8; 3741) & 832.846$^{***}$ (df = 3; 3746) \\ 
\hline 
\hline \\[-1.8ex] 
\textit{Note:}  & \multicolumn{2}{r}{$^{*}$p$<$0.1; $^{**}$p$<$0.05; $^{***}$p$<$0.01} \\ 
\end{tabular} 
\end{table} 

\newpage

\subsection*{Heterogeneity}

% Table created by stargazer v.5.2.2 by Marek Hlavac, Harvard University. E-mail: hlavac at fas.harvard.edu
% Date and time: ju., jul. 04, 2019 - 19:21:12
\begin{table}[!htbp] \centering 
  \caption{Total agricultural production in confederate states including population growth} 
  \label{tab:agoutcsa} 
\begin{tabular}{@{\extracolsep{5pt}}lcc} 
\\[-1.8ex]\hline 
\hline \\[-1.8ex] 
 & \multicolumn{2}{c}{\textit{Dependent variable:}} \\ 
\cline{2-3} 
\\[-1.8ex] & \multicolumn{2}{c}{Aggregate agricultural output (logs)} \\ 
\\[-1.8ex] & \textit{OLS} & \textit{FE} \\ 
\\[-1.8ex] & (1) & (2)\\ 
\hline \\[-1.8ex] 
 Slave ratio & 1.337$^{***}$ &  \\ 
  & p = 0.000 &  \\ 
  Before the 13th Amendment & 0.047$^{*}$ & 0.066$^{***}$ \\ 
  & p = 0.069 & p = 0.0002 \\ 
  Diff-in-diff estimator & 0.016 & $-$0.530$^{***}$ \\ 
  & p = 0.825 & p = 0.000 \\ 
  Implements and machinery (logs) & 0.319$^{***}$ & 0.147$^{***}$ \\ 
  & p = 0.000 & p = 0.000 \\ 
  Livestock (logs) & 0.349$^{***}$ & 0.336$^{***}$ \\ 
  & p = 0.000 & p = 0.000 \\ 
  Number of farms (logs) & 0.349$^{***}$ & 0.332$^{***}$ \\ 
  & p = 0.000 & p = 0.000 \\ 
  Foreign people ratio & $-$0.767$^{***}$ & 0.521 \\ 
  & p = 0.005 & p = 0.144 \\ 
  Population growth rate 1860-1870 & $-$0.020 &  \\ 
  & p = 0.505 &  \\ 
  Constant & 2.547$^{***}$ &  \\ 
  & p = 0.000 &  \\ 
 \hline \\[-1.8ex] 
Observations & 2,035 & 2,035 \\ 
R$^{2}$ & 0.919 & 0.749 \\ 
Adjusted R$^{2}$ & 0.918 & 0.478 \\ 
F Statistic &  & 486.082$^{***}$ (df = 6; 977) \\ 
\hline 
\hline \\[-1.8ex] 
\textit{Note:}  & \multicolumn{2}{r}{$^{*}$p$<$0.1; $^{**}$p$<$0.05; $^{***}$p$<$0.01} \\ 
\end{tabular} 
\end{table} 

% Table created by stargazer v.5.2.2 by Marek Hlavac, Harvard University. E-mail: hlavac at fas.harvard.edu
% Date and time: ju., jul. 04, 2019 - 19:21:13
\begin{table}[!htbp] \centering 
  \caption{Cotton production per total agricultural production in confederate states including population growth} 
  \label{tab:cotcsa} 
\begin{tabular}{@{\extracolsep{5pt}}lcc} 
\\[-1.8ex]\hline 
\hline \\[-1.8ex] 
 & \multicolumn{2}{c}{\textit{Dependent variable:}} \\ 
\cline{2-3} 
\\[-1.8ex] & \multicolumn{2}{c}{Cotton ratio} \\ 
\\[-1.8ex] & \textit{OLS} & \textit{FE} \\ 
\\[-1.8ex] & (1) & (2)\\ 
\hline \\[-1.8ex] 
 Slave ratio & 0.115$^{***}$ &  \\ 
  & p = 0.006 &  \\ 
  Before the 13th Amendment & $-$0.069$^{***}$ & $-$0.001 \\ 
  & p = 0.000 & p = 0.790 \\ 
  Diff-in-diff estimator & 0.616$^{***}$ & $-$0.107$^{***}$ \\ 
  & p = 0.000 & p = 0.00000 \\ 
  Implements and machinery (logs) & $-$0.027$^{***}$ & 0.029$^{***}$ \\ 
  & p = 0.001 & p = 0.00000 \\ 
  Livestoc (logs)k & 0.008 & $-$0.0005 \\ 
  & p = 0.207 & p = 0.957 \\ 
  Number of farms (logs) & 0.009 & $-$0.024$^{***}$ \\ 
  & p = 0.152 & p = 0.002 \\ 
  Foreign people ratio & $-$0.046$^{*}$ & $-$0.049 \\ 
  & p = 0.069 & p = 0.586 \\ 
  Population growth rate 1860-1870 & 0.002 &  \\ 
  & p = 0.805 &  \\ 
  Constant & 0.140$^{***}$ &  \\ 
  & p = 0.0002 &  \\ 
 \hline \\[-1.8ex] 
Observations & 2,035 & 2,035 \\ 
R$^{2}$ & 0.694 & 0.279 \\ 
Adjusted R$^{2}$ & 0.693 & $-$0.500 \\ 
F Statistic &  & 63.136$^{***}$ (df = 6; 977) \\ 
\hline 
\hline \\[-1.8ex] 
\textit{Note:}  & \multicolumn{2}{r}{$^{*}$p$<$0.1; $^{**}$p$<$0.05; $^{***}$p$<$0.01} \\ 
\end{tabular} 
\end{table} 

% Table created by stargazer v.5.2.2 by Marek Hlavac, Harvard University. E-mail: hlavac at fas.harvard.edu
% Date and time: ju., jul. 04, 2019 - 19:21:14
\begin{table}[!htbp] \centering 
  \caption{Tobacco production per total agricultural production in confederate states including population growth} 
  \label{tab:tobcsa} 
\begin{tabular}{@{\extracolsep{5pt}}lcc} 
\\[-1.8ex]\hline 
\hline \\[-1.8ex] 
 & \multicolumn{2}{c}{\textit{Dependent variable:}} \\ 
\cline{2-3} 
\\[-1.8ex] & \multicolumn{2}{c}{Tobacco ratio} \\ 
\\[-1.8ex] & \textit{OLS} & \textit{FE} \\ 
\\[-1.8ex] & (1) & (2)\\ 
\hline \\[-1.8ex] 
 Slave ratio & $-$0.002$^{***}$ &  \\ 
  & p = 0.00000 &  \\ 
  Before the 13th Amendment & $-$0.0001 & 0.002 \\ 
  & p = 0.230 & p = 0.362 \\ 
  Diff-in-diff estimator & 0.0004$^{*}$ & $-$0.026$^{***}$ \\ 
  & p = 0.079 & p = 0.001 \\ 
  Implements and machinery (logs) & 0.0001$^{***}$ & $-$0.003 \\ 
  & p = 0.00001 & p = 0.218 \\ 
  Livestock (logs) & $-$0.0001$^{***}$ & 0.004 \\ 
  & p = 0.004 & p = 0.285 \\ 
  Number of farms (logs) & 0.0001$^{***}$ & $-$0.002 \\ 
  & p = 0.0004 & p = 0.591 \\ 
  Foreign people ratio & $-$0.001$^{***}$ & $-$0.017 \\ 
  & p = 0.0005 & p = 0.622 \\ 
  Population growth rate 1860-1870 & $-$0.0001$^{*}$ &  \\ 
  & p = 0.056 &  \\ 
  Constant & 0.0002 &  \\ 
  & p = 0.472 &  \\ 
 \hline \\[-1.8ex] 
Observations & 2,035 & 2,035 \\ 
R$^{2}$ & 0.269 & 0.033 \\ 
Adjusted R$^{2}$ & 0.266 & $-$1.013 \\ 
F Statistic &  & 5.541$^{***}$ (df = 6; 977) \\ 
\hline 
\hline \\[-1.8ex] 
\textit{Note:}  & \multicolumn{2}{r}{$^{*}$p$<$0.1; $^{**}$p$<$0.05; $^{***}$p$<$0.01} \\ 
\end{tabular} 
\end{table} 


%
%
\end{document}
